\documentclass[addpoints, 12pt] {exam}
\usepackage{graphicx}
\usepackage{amsmath}
\bracketedpoints
\pagestyle{headandfoot}
\runningheadrule
\firstpageheader{Math 112}{Written Homework 8}{Due March 15th 2024}
\runningheader{Math 112}{ Page\; \thepage\; of\; \numpages}{Written Homework 8}
\firstpagefooter{}{}{}
\runningfooter{}{}{}
\setlength\answerskip{2ex}
\setlength\answerlinelength{1.5in}
\begin{document}

\begin{center}
\fbox{\fbox{\parbox{5.5in}{\centering
Directions:\\Please only put your final, well written solutions, in the space provided.\\ Give exact answers (simplified radicals or fractions).\\If you use additional paper clearly label the question and upload pages after the question page.\\Use complete sentences and explain your reason as much as possible.\\There are \numquestions\,  questions and \numpoints\, points total
}}}\end{center}
\vspace{0.1in}
\makebox[\textwidth]{Name:\enspace\hrulefill}
%\qformat{Question \thequestion \dotfill \thepoints}%

\begin{questions}
\question Answer each question as fully as possible, give examples or explanations in your own words.
\begin{parts}
\part[4] List two different forms for a quadratic function. You may use specifical numerical values, but you need to label what each value {\bf means} in the context of the formula, function, or graph of the function.
%Expected answers are the factored form and the quadratic vertex form, but there also may be the standard form.%
\vspace{2in}
\part[4] Suppose you are asked to find approximate solutions to the equation \(2x^2 - 3x -7=0\). This cannot be easily factored, so you will need to use the {\bf quadratic formula}. How do you plan to use the formula? E.g., do you have it saved on a calculator? Do you have it memorized? Give this some thought and make sure you are prepared.
\vspace{2in}
\part[2] What are the {\bf approximate} solutions to the equation in the previous part? Use the method you describe above.\answerline
%-1.266 and 2.766%
\end{parts}
\newpage
\question A motorcycle stunt rider jumped across the Snake River, starting from a bluff overlooking the river and landing on a platform at the level of the river. The height of her jump and position relative to the launch ramp were recorded. The height, as a function of the horizontal distance from the ramp, was modeled using the function \(H(d) = -0.0003d^2+2.19d + 350\), where \(H\) is measured in feet above the river and \(d\) is measured in feet from the ramp.
\begin{parts}
\part[3] Sketch a model of this scenario, based on the description provided.\vspace{2in}
\part[2]How high above the river was the rider when she left the launch ramp?\vspace{0.1in}\answerline
\part[2]What was the riders's maximum height above the river?\vspace{1in}\answerline
\part[2]How far from the ramp (horizontally) was she when she was at her max height?\answerline
\part[1]How far from the ramp (horizontally) was she when she landed on the pad?\answerline
\end{parts}
\newpage
\question A guitar rental company estimates that for each $\$20$ increase in monthly rental price, the number of people who rent guitars drops by 1. The current rental price is \(\$80\) per month and there are \(30\) guitars rented. 
\begin{parts}
\part[1] Given their current rental price, how much are they earning each month? 
% 80*30 = 2400%
\answerline
\part[2] If {\bf increase} their rentals by \(\$60\) (for a total monthly cost of \(\$140\)) how many guitars can they expect to rent? 
%60/20 = 3, 30 - 3 = 27%
\answerline
\part[2] If they charge \(\$140\), how much would they estimate their earnings per month to be? 
% (140)*27=3780%
\answerline
\part[4] Suppose that the company plans to rent their guitars at \(x\) dollars per more (or less) than their current \(\$80\) plan. If their total estimated monthly rental earnings is represented by \(R\), determine the equation that represents \(R\) as a function of \(x\).
%R(x) = (80+x)(30 - x/20)
\vspace{1in}\answerline
\part[1] What monthly rental price {\bf maximizes} the possible monthly earnings?  What are the maximum earnings predicted?\vspace{1in}\answerline
%80+260 = 340, R(260) = 5780%
\end{parts}
\newpage
\question A furniture manufacturing company recently completed some market research and have determined they can model the price/demand for a new line of couches with the following: \(p = 3750 -0.25q\) where \(p\) is the retail price of the couch and \(q\) are the number of couches demanded.
\begin{parts}
\part[2]What is the revenue as a function of number of couches sold?\vspace{0.5in}\answerline
\part[2]The fixed costs associated with retooling the manufacturing plant is \(\$500,000\) and once the factory is operational, each couch will cost incur an additional \(\$425\) to make. Based on this information, what is the cost as a function of the number of couches manufactured?\vspace{0.5in}\answerline
\part[2]If the number of couches sold is exactly the number of couches manufactured, what is the profit as a function of the number of couches sold?\vspace{0.5in}\answerline
\part[2]What are the break-even points for the profit (reminder: \emph{break-even} means the profit is exactly \(\$0\)). Round to the nearest whole number.\vspace{1in}\answerline\newpage Question 4 cont'd
\part[1]How many couches should be manufactured/sold in order to maximize profit? Round to the nearest whole number.\vspace{3in}\answerline
\part[1]How much profit can the company anticipate if they maximize their profit? Round to the nearest million dollars. 
\end{parts}
\end{questions}


\end{document}