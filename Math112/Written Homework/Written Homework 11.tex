\documentclass[addpoints, 12pt] {exam}
\usepackage{graphicx}
\usepackage{amsmath}
\bracketedpoints
\pagestyle{headandfoot}
\runningheadrule
\firstpageheader{Math 112}{Written Homework 11}{Due April 5th 2024}
\runningheader{Math 112}{ Page\; \thepage\; of\; \numpages}{Written Homework 11}
\firstpagefooter{}{}{}
\runningfooter{}{}{}
\setlength\answerskip{2ex}
\setlength\answerlinelength{1.5in}
\begin{document}

\begin{center}
\fbox{\fbox{\parbox{5.5in}{\centering
Directions:\\Please only put your final, well written solutions, in the space provided.\\ Give exact answers (simplified radicals or fractions).\\If you use additional paper clearly label the question and upload pages after the question page.\\Use complete sentences and explain your reason as much as possible.\\There are \numquestions\,  questions and \numpoints\, points total
}}}\end{center}
\vspace{0.1in}
\makebox[\textwidth]{Name:\enspace\hrulefill}
%\qformat{Question \thequestion \dotfill \thepoints}%

\begin{questions}

\question For this question, we will be using the function \[\displaystyle h(x) = 4\left(0.56\right)^{x+6}-9\]
\begin{parts}
\part[2] What is the {\bf domain} of this function? Please give your answer in interval notation.\vspace{0.5in}\answerline
\part[2] What is the {\bf range} of this function? Please give your answer in interval notation.\vspace{0.5in}\answerline
%%Students don't have logs yet fix this part
\part[2] What are the {\bf zeros} of \(g(x)\)? \vspace{0.5in} \answerline
\part[1] If there is one, what is the {\bf horizontal asymptote}?  \newline Write DNE if it doesn't exist.\answerline
\part[1] If it exists, what is the {\bf vertical intercept}? \newline Write DNE if it doesn't exist.\answerline
\part[2] Draw and label the graph of the function based on your findings above. Be sure to include and clearly label each of the features you found.
\end{parts}
\newpage
\question Pamela is looking ahead and planning their retirement. They have a certain account which is expected to earn \(6.1\%\) interest {\bf compounded continously}.  Use a graphing calculator to answer the questions below.
\begin{parts}
\part[2] If Pamela invests \(P\) dollars, what will her future earnings \(A\) be as a function of time in years, \(t\)? \answerline
\part[3] How many years will it take to {\bf triple} Pamela's initial investment?\vspace{1in}\answerline
\part[3] Suppose that Pamela's desired retirement would require the account to have \(\$500,000\) after 20 years. How much, rounded to the nearest {\bf thousand}, would they need to invest into this account?\vspace{1in}\answerline
\part[2] Let's suppose that Pamela only has \(\$1000\) to invest and only makes the one investment. How many {\bf years} will it take for her account to have \(\$500,000\) in it? \vspace{1in}\answerline
\end{parts}
\newpage
\question A car with an initial value of \(\$43,000\) will depreciate exponentially as the years go by. After 5 years, the car was only worth \(\$19,000\). Let \(A(t)\) represent the value of the car \(t\) years after its initial purchase.
\begin{parts}
\part[3] Determine \emph{an} exponential model for this car (you can use any of the forms).\vspace{1.5in}\answerline
\part[2] Using your model, what will the car be worth 10 years after its initial purhcase?\answerline
\part[3] Using a graphing utility, when will the car be worth \(\$10,000\)? Round your answer to the nearest year.\answerline
\part[2] What is the annual depreciation rate? (You will need the \(A(t)=P(1+r)^t\) form where \(r\) is the rate)\vspace{1.5in}
\end{parts}
\newpage
\question Scientists discovered a mineral with radioactive embedded within. After extracting the radioactive isotopes and monitoring the mass of the material, after just 20 days the mass was \emph{half} of what it was when they first extracted it.
\begin{parts}
\part[5] Given that the initial mass was 100g, express the mass of the sample as a function of the number of days since it was initially observed (this will be an exponential model).
\vspace{2in}\answerline
\part[5] Given the amount of radioactivity, scientists have determined the material will be considered safe and can be viewed without careful shielding so long as there is no more than \(12.5\) grams of the material remaining. How many days will they need to wait (after the initial mass was discovered)? Please explain/show/include a screenshot of the graph to indicate how you found your solution.
\vspace{2in}\answerline
\end{parts}
\end{questions}


\end{document}