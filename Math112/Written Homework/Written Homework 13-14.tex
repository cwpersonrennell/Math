\documentclass[addpoints, 12pt] {exam}
\usepackage{graphicx}
\usepackage{amsmath}
\bracketedpoints
\pagestyle{headandfoot}
\runningheadrule
\firstpageheader{Math 112}{Written Homework 13/14}{Due April 19th 2024}
\runningheader{Math 112}{ Page\; \thepage\; of\; \numpages}{Written Homework 13/14}
\firstpagefooter{}{}{}
\runningfooter{}{}{}
\setlength\answerskip{2ex}
\setlength\answerlinelength{1.5in}
\begin{document}

\begin{center}
\fbox{\fbox{\parbox{5.5in}{\centering
Directions:\\Please only put your final, well written solutions, in the space provided.\\ Give exact answers (simplified radicals or fractions).\\If you use additional paper clearly label the question and upload pages after the question page.\\Use complete sentences and explain your reason as much as possible.\\There are \numquestions\,  questions and \numpoints\, points total
}}}\end{center}
\vspace{0.1in}
\makebox[\textwidth]{Name:\enspace\hrulefill}
%\qformat{Question \thequestion \dotfill \thepoints}%

\begin{questions}

\question {\bf Logarithms} \newline Part 1: For the first part of this question, we will be using the function\[y=f(x) = -\log_3{\left(4x+7\right)} + 2\]
\begin{parts}
\part[1] What is the domain of the function? Give your answer using interval notation.\vspace{0.25in} \answerline
\part[1] What is the range of the function? Give you answer using interval notation. \vspace{0.25in}\answerline
\part[1] What, if any, are the horizontal intercept(s)? \vspace{0.25in}\answerline
\part[1] What is the vertical asymptote of the function?\answerline
\part[1] Given the function as defined above, \(y=f(x)\), find \(f^{-1}(y)\).\vspace{0.5 in}\answerline
\end{parts}\newpage
{\bf Logarithms} \newline
Part 2: Condensing logarithmic expression
\begin{parts}
\part[5] Write \[3\log_2 x+\frac{1}{5}\log_2 y - 4\log_2 z\] in the form \(\log_b P\) where \(P\) is an algebraic expression. Put another way, write the expression as a single logarithm by combining the logarithms using the properties of logs. 
\vspace{5in}\answerline
\end{parts}
\newpage
\question {\bf Solving Exponential and Logarithmic Equations}\newline
Solve each of the equations below for the unknown variable. {\bf Show your work}
\begin{parts}
\part[4] \(\log_7(y+8)=\log_7(y-1)+\log_7(y)\)\vspace{2in}\answerline
\part[3] \(\log_{12}(1-y) + \log_{12}(-y) = 1\)\vspace{2in}\answerline
\part[3] \(\displaystyle\frac{100}{1+e^{0.05x}} = 4\)\vspace{2in}\answerline
\end{parts}\newpage
\question {\bf Application Questions}
\begin{parts}
\part[5]The number of AIDS-related deaths worldwide has been decreasing exponentially over the last few decades.  The World Health Organization (WHO) estimates that 2.3 million people died from AIDS in the year 2007, and 1.5 million people died from AIDS in the year 2016. 

Use the given data to create an exponential decay function in the form of $y=Pe^{rt}$, where $t$ is the number of years since 2007, the unit of $y$ is **million**, and $r$ is rounded to **two decimals**.
\vspace{2in}\answerline

\part[5] A scientist has 100 grams of a radioactive substance. The amount of time, $T$, in days that it takes the radioactive substance to decay to $x$ grams is given by $T=f(x)=-50 \ln\frac{x}{100}$. How many grams of the radioactive substance will there be after 10 days? Round your answer to the **nearest gram**.

Tip: there are two ways to do this problem. You can either use the information to set up an equation or first find $f^{-1}(x)$ and then evaluate $f^{-1}(10)$.
\vspace{2in}\answerline
\end{parts}
\newpage\question {\bf Applications of Money and Radiation}
\begin{parts}
\part[5]Suppose Ali initially invests $\$4000$ in an account bearing $4.8\%$ interest compounded monthly. How many years will it take for the deposit to triple in value? Round to the **nearest 0.01 year**.
\vspace{2in}\answerline
\part[5] Suppose 100mg of a radioactive substance exponentially decays to 20mg in 8 days. What is the half-life of the substance? Round to the **nearest day**. 
\vspace{2in}\answerline
\end{parts}
\end{questions}


\end{document}