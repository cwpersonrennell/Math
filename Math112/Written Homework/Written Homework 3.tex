\documentclass[addpoints, 12pt] {exam}
\usepackage{graphicx}
\usepackage{amsmath}
\bracketedpoints
\pagestyle{headandfoot}
\runningheadrule
\firstpageheader{Math 112}{Written Homework 3}{Due November 9th 2023}
\runningheader{Math 112}{ Page\; \thepage\; of\; \numpages}{Written Homework 3}
\firstpagefooter{}{}{}
\runningfooter{}{}{}
\setlength\answerskip{2ex}
\setlength\answerlinelength{1.5in}
\begin{document}

\begin{center}
\fbox{\fbox{\parbox{5.5in}{\centering
Directions:\\Please only put your final, well written solutions, in the space provided.\\ Give exact answers (simplified radicals or fractions).\\If you use additional paper clearly label the question and upload pages after the question page.\\Use complete sentences and explain your reason as much as possible.\\There are \numquestions\,  questions and \numpoints\, points total
}}}\end{center}
\vspace{0.1in}
\makebox[\textwidth]{Name:\enspace\hrulefill}
%\qformat{Question \thequestion \dotfill \thepoints}%

\begin{questions}
\question There are a few different equations for lines, but there are two that are especially useful:
\[
\begin{array}{lrcl}
\text{Slope-Intercept:}& y&=&mx+b\\
\text{Point-Slope:}&y-y_1&=&m(x-x_1)\\
\text{Point-Slope (alt):}&y&=&m(x-x_1)+y_1
\end{array}
\]
Note that I put \textbf{3} equations, but in reality the last two equations are the same, but the third equation is what you get when you solve for \(y\) (which is quite useful in most circumstances).
\begin{parts}
\part[3] A line has a slope of \(\frac{3}{4}\) and a vertical intercept of \(0\). Write the equation of the line in \emph{Slope-Intercept} form.\answerline
\part[3] A line passes through two points: \((1,1)\) and \((5,-3)\). Write the equation of the line in \emph{Point-Slope} form.\vspace{1in}\answerline
\part Given the equation \(y=-\frac{x}{4}+3\)\begin{subparts}
\subpart[1] What is the slope?\answerline
\subpart[1] What is the vertical intercept?\answerline
\end{subparts}
\part[2] Given the equation \(y=3(x-4)+7\), state the point the line passes through and is represented in the equation.\answerline
\end{parts}
\newpage
\question A company is planning an end-of-year party at the local convention center. Normally, there is a flat \(\$2500\) booking fee and a \(\$20.00\) per person charge. 
\begin{parts}
\part[3] Using the fee structure described above, express the total cost for the party as a function of the number of people attending.\vspace{0.5in}\answerline
\part[1]While the previous cost is the typical fee structure, unfortunately this year the per-person fee has increased by \(20\%\). Express the total cost for the party as a function of the number of people attending, taking the cost increase into consideration.\vspace{0.5in}\answerline
\part[3] Using a \emph{graphing calculator}, determine how many people can attend the party (with the cost increase) if the budget is \(\$10,000\). Round up to the nearest whole number. \answerline
\part[3] The convention center needs to prepare the space prior to the party and clean the space aftewards. They will need to hire/pay personnel at \(\$160\) per person per day. If they take two days to prep/clean and have a staff of 25 people, what is the total \emph{profit} the convention center will earn if the company spends the full \(\$10,000\)?

\end{parts}\newpage
\question For this question, we will be considering a \emph{simple} repayment of a loan. Suppose that Robert is paying off a \(\$2500\) debt by making equal monthly payments over the course of 12 months at which point the debt will be fully repaid.
\begin{parts}
\part[3] Represent the total amount of the loan \emph{remaining} as a function of the time in months.\vspace{0.5in}\answerline
\part[1] What is the domain of this function?\answerline
\part[1] What is the range of this function?\answerline
\part[1] What is the slope of this function?\answerline
\part[1] What is a practical interpretation of the slope of the line?\vspace{1in}
\part[1] Where is the function increasing?\answerline
\part[1] Where is the function decreasing?\answerline
\part[1] Sketch the line using the full domain and range. You may use a separate page if you wish.
\end{parts}\newpage
\question A manufacturing plant produces bamboo plates. To initiate the manufacturing process, there is a fixed cost of \(\$12,000\) to get the manufactoring tools in place and a per-plate cost of \(\$0.12\). 
\begin{parts}
\part[3] What is the total cost as a function of plates manufactured? \vspace{0.5in}\answerline
\part[1] The company plans to sell the plates for \(\$2.99\) per plate.
What is the revenue as a function of the number plates sold?\vspace{0.25in}
\answerline
\part[3] \emph{Assuming} that the company sells as many plates as they manufacture, what is their profit as a function of the number of plates sold/made? (Profit is Revenue less Cost)\vspace{0.5in}\answerline
\part[3] How many plates will the company need to sell/make in order to break-even? (Breaking Even means they have \(0\) Profit).

\end{parts}
\end{questions}


\end{document}