\documentclass[addpoints, 24pt] {exam}
\usepackage{graphicx}
\usepackage{amsmath}
\usepackage{xcolor}
\bracketedpoints

\setlength\answerskip{2ex}
\setlength\answerlinelength{1.5in}
{\def\dfrac#1#2{\displaystyle\frac{#1}{#2}}

{\def\solve#1{\begin{array}{rcl}{#1}\end{array}}}

\begin{document}
Find an exact simplified solution to the equation in the interval \(0<x<1\).
\[
1 = 12\cos(x+1)-5
\]
\newpage
Find the solutions to the equation in the interval \(0\leq \beta \leq \pi\).
\[
2\cos(2\beta)+1=0
\]
\newpage
Simplify the trigonometric expression
\[
\displaystyle \frac{2+\cot^2(x)}{\csc^2(x)}-1
\]\newpage
Make the indicated trigonometric substitution in the given algebraic expression and simplify 

(see Example 7). Assume that \(0<\theta<	\frac{\pi}{2}\)
\[
\begin{array}{rcr}
\displaystyle \frac{\sqrt{x^2-4}}{x}&,&x=2\sec(\theta)
\end{array}
\]
\newpage
\[
\color{red}
f(x) = x^2-3
\]
\[
T(x) = \sqrt{x+12} -5
\]
\[\begin{array}{rcl}
(T\circ f)(x) &=& T({\color{red}f(x)})\\
(T\circ f)(x) &=&T({\color{red}x^2-3})\\
(T\circ f)(x) &=&\sqrt{{\color{red}x^2-3}+12}-5\\
(T\circ f)(x) &=&\sqrt{x^2+9}-5
\end{array}
\]
\newpage
\[
\begin{array}{rcl}
a^2+b^2 &=& c^2\\
b^2 &=&c^2-a^2\\\
b&=&\pm\sqrt{c^2-a^2}
\end{array}
\]
\[
x(x^3+27)=x(x+3)(x^2-3x+9)
\]
\newpage
\[
\begin{array}{rcl}
3x^{3/2}-12x^{1/2}+9x^{-1/2}&=&3x^{-1/2}\left(x^2-4x+3\right)\\
&=&3x^{-1/2}(x-3)(x-1)
\end{array}
\]
\[
f(x) =\sqrt[4]{x}
\]
\[
f(-x) = \sqrt[4]{-x}
\]

\[
x^4 - 10x^2 +9 = 0
\]
Replace the variable to reduce the overall degree of the problem. Here is the best choice for that:
\[x^2 = y\]
\[
y^2-10y+9 = 0\]
Once you solve for \(y\), don't forget you need to actually solve for \(x\), so go back to \(x^2=y\) and replace \(y\) with all the solutions you find. You may have up to 4 solutions for \(x\).
\[
	\begin{array}{|c|c|c|c|c|}
		\hline
		(-\infty,-2] & [-2,1-\sqrt{5}] & [1-\sqrt{5},2] & [2,1+\sqrt{5}] & [1+\sqrt{5},\infty) \\\hline
		\times & \times & \times & Yo & \times \\\hline
	\end{array}
\]
\newline
\large
$$\begin{array}{rcl}
\dfrac{5x}{6}-\dfrac{\pi}{6}&=&\dfrac{1}{6}\left(5x-\pi\right)\\\\
&=&\dfrac{1}{6}\times 5\left(x-\frac{\pi}{5}\right)\\\\
&=&\dfrac{5}{6}\left(x-\frac{\pi}{5}\right)
\end{array}$$

$$\begin{array}{rcl}
\dfrac{3x}{4}-\dfrac{\pi}{6}&=&\dfrac{9x}{12}-\dfrac{2\pi}{12}\\\\
&=&\dfrac{1}{12}\left(9x-2\pi\right)\\\\
&=&\dfrac{1}{12}\times 9\left(x-\frac{2\pi}{9}\right)\\\\
&=&\dfrac{3}{4}\left(x-\frac{2\pi}{9}\right)
\end{array}$$

\[
\begin{array}{rcl}
\log_5\left(\dfrac{\sqrt{5x^9}}{y}\right) &=&\log_5\left(\dfrac{\sqrt{5}\cdot \sqrt{x^9}}{y}\right)\\\\&=&\log_5(\sqrt{5})+\log_5\left(\sqrt{x^9}\right)-\log_5(y)\\\\ &=& \log_5(5^{1/2}) + \log_5(x^{9/2}) - \log_5(y)\\\\
&=&\frac{1}{2}\log_5(5)+\frac{9}{2}\log_5(x)-\log_5(y)\\\\
&=&\frac{1}{2}+\frac{9}{2}\log_5(x)-\log_5(y)
\end{array}
\]

\[ 
\begin{array}{rcl} 
\text{Power Rule}&:&\log_B(A^n)=n\log_B(A)\\\\ 
\text{Product Rule}&:&\log_B(A\times C) = \log_B(A)+\log_B(C)\\\\
\text{Quotient Rule}&:&\log_B\left(\frac{A}{C}\right)=\log_B(A)-\log_B(C) 
\end{array}
\]

\[
\begin{array}{rcl}
\log_{12}(9)+2\log_{12}(4) &=&\log_{12}(9)+\log_{12}(4^2)\\
\log_{12}(144) &=& \log_{12}(12^2) = 2
\end{array}
\]

\[
\ln e^7-\ln e^2 = 7\ln e - 2\ln e = 7-2
\]

Recall the main definition of Logarithm:
\[
\log_B{x}=y\leftrightarrow x=B^y
\]
Then rearrange to get the Logarithm alone so you can use the definition:
\[
\begin{array}{rcl}
y&=&-\log_3(4x+7)+2\\\\
y-2&=&-\log_3(4x+7)\\\\
-y+2&=&\log_3(4x+7)\\\\
3^{-y+2}&=&4x+7\\\\
3^{-y+2}-7&=&4x\\\\
\dfrac{3^{-y+2}-7}{4}&=&x\\\\
f^{-1}(y)&=&\dfrac{3^{-y+2}-7}{4}
\end{array}
\]


\end{document}