\documentclass[addpoints, 12pt] {exam}
\usepackage{graphicx}
\usepackage{amsmath}
\bracketedpoints
\pagestyle{headandfoot}
\runningheadrule
\firstpageheader{Math 112}{Written Homework 0}{Due January 12th 2024}
\runningheader{Math 112}{ Page\; \thepage\; of\; \numpages}{Written Homework 0}
\firstpagefooter{}{}{}
\runningfooter{}{}{}
\setlength\answerskip{2ex}
\setlength\answerlinelength{1.5in}
\begin{document}

\begin{center}
\fbox{\fbox{\parbox{5.5in}{\centering
Directions:\\Please only put your final, well written solutions, in the space provided.\\ Give exact answers (simplified radicals or fractions).\\If you use additional paper clearly label the question and upload pages after the question page.\\Use complete sentences and explain your reason as much as possible.\\There are \numquestions\,  questions and \numpoints\, points total
}}}\end{center}
\vspace{0.1in}
\makebox[\textwidth]{Name:\enspace\hrulefill}
%\qformat{Question \thequestion \dotfill \thepoints}%

\begin{questions}
\question Answer each question below. You will receive feedback on your work, so please  demonstrate all your steps and provide your detailed thoughts when requested.
\begin{parts}
\part[5] Clearly state the Order of Operations and explain how it works and what it means in your own words.
\vspace{2.25in}
\part[5] Simplify the expression according to the order of operations. Please show every step (only do one step at a time). Your \textbf{final} answer should go on the line provided. \[\left(-3\right)^2-9(15-7)\]
\vspace{1.75in}\answerline
\end{parts}
\newpage
\question Answer each question and provide as much detail to your answer as possible. You will receive feedback on all of your work. 

\textbf{For each part: Provide a written explanation AND an example problem of your own creation to demonstrate your understanding. Ask your instructor any questions you have for clarification about this problem.}
\begin{parts}
\part[2] In your own words, if you are \textbf{adding or subtracting} two fractions, what are the rules? \vspace{3in}
\part[2] In your own words, if you are \textbf{multiplying} two fractions, what are the rules? \newpage
\part[2]  In your own words, if you are \textbf{dividing} two fractions, what are the rules? \vspace{2.2in}
\part[2] In your own words, if you are asked to \textbf{simplify} a fraction, what are the rules?  \vspace{2.2in}
\part[2] In your own words, if you need to simplify a fraction that has \textbf{other fractions} inside it (this is known as a \textbf{compound fraction}) what steps would you take? \newpage
\end{parts}
\newpage
\question \textbf{Simplify each expression.} Show all your steps when answering each question. Put your final answer in the space provided.
\begin{parts}
\part[2] \(-5\left(y-3\right)\)\vspace{0.5in}\answerline
\part[2] \(\left(-5+w\right)\left(w-3\right)\)\vspace{1in}\answerline
\part[2] \(\left(-2x+y\right)\left(\displaystyle\frac{3}{2}\right)\)\vspace{1in}\answerline
\part[2] \(\displaystyle\frac{x(x+6)(x-6)}{x(x-6)}\)\vspace{1in}\answerline
\part[2] \(\displaystyle\frac{1-x^2}{2+x}\)\vspace{0.75in}\answerline
\end{parts}
\newpage
\question This question is really for you. Please consider thoughtfully before answering. Introspection and careful consideration is important and I am interested in understanding your motivations, but moreover, I am hoping you will also come to understand yourself a tiny bit better too.
\begin{parts}
\part[2] What is the most \textbf{obvious} reason you are taking this course?\vspace{4in}
\part[2] Beyond the reason you stated above, why are you taking this course?\vspace{4in}\newpage
\part[2] What do you think is the best way to measure your \textbf{progress} for this class?\vspace{4in}
\part[2] What does it mean for you to be \textbf{successful} in this class? \vspace{4in}\newpage
\part[2] The semester has just begun, do you have any personal questions, comments, or thoughts you would like to share as we begin this journey together?

\end{parts}
\end{questions}


\end{document}