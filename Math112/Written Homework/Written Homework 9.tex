\documentclass[addpoints, 12pt] {exam}
\usepackage{graphicx}
\usepackage{amsmath}
\bracketedpoints
\pagestyle{headandfoot}
\runningheadrule
\firstpageheader{Math 112}{Written Homework 9}{Due March 18th 2024}
\runningheader{Math 112}{ Page\; \thepage\; of\; \numpages}{Written Homework 9}
\firstpagefooter{}{}{}
\runningfooter{}{}{}
\setlength\answerskip{2ex}
\setlength\answerlinelength{1.5in}
\begin{document}

\begin{center}
\fbox{\fbox{\parbox{5.5in}{\centering
Directions:\\Please only put your final, well written solutions, in the space provided.\\ Give exact answers (simplified radicals or fractions).\\If you use additional paper clearly label the question and upload pages after the question page.\\Use complete sentences and explain your reason as much as possible.\\There are \numquestions\,  questions and \numpoints\, points total
}}}\end{center}
\vspace{0.1in}
\makebox[\textwidth]{Name:\enspace\hrulefill}
%\qformat{Question \thequestion \dotfill \thepoints}%

\begin{questions}
\question
\begin{parts}
\part[1] Define, in general terms, what the \emph{domain} of a function is.\vspace{0.5in}
\part[1] Define, in gernal terms, the \emph{range} of a function.\vspace{0.5in}
\part[2] What is the mathematical definition of a function\vspace{1in}
\part[2] Write *one* of the forms for the equation of a line. \answerline
\part[2] In your own words, what is a polynomial?\vspace{1in}
\part[2] What is the \emph{degree} of the polynomial function: \[f(x)=-3+2x+x^2-\frac{2}{3}x^3\]\answerline

\end{parts}
\newpage
\question Various businesses, banks first and foremost, but also investment firms, will accept money from you and offer to pay you for that money by paying it back after a fixed time period with interest. Interest rates depending on {\bf many} factors. Typically, this interest is compounded based on the type of account or investment you make. For the purposes of this problem, let us suppose that you have \(\$1000\) to deposit into a bank. After \(10\) years, the amount of money that would be in the account is given by the formula 
\[A=1000(1+r)^{10}\]where \(A\) is the amount in the account and \(r\) is the annual interest rate. 
\begin{parts}
\part[2] Provide an explanation of how the function above compares with the base function \(f(r)=r^{10}\). Use terminology from this course, such how \(f\) can be transformed into \(A\). \vspace{0.5in}
\part[2] Give a {\bf reasonable} set of bounds on the possible interest rates. Put another way, while the technical domain is not restricted, {\bf practically} the interest rate cannot be infinite. Most answers are acceptable {\bf as long as you provide an explanation}.\vspace{0.5in}\answerline
\part[6] If you wanted to {\bf double} your initial investment in 10 years, what interest rate would you need to secure?\vspace{.5in}\answerline
\end{parts}
\newpage
\question A package can only be sent by mail if the sum of its height and the perimeter of its base is no more than 96 inches. Suppose we are mailing a package with a \emph{square} base and plan to use *all* 96 inches available.
\begin{parts}
\part[3] Sketch a drawing of the package. Choose labels/variables for each of the sides and the height.\vspace{2in}
\part[1] What is the perimeter of the base, using the variables from your drawing.\answerline
\part[1] Write the equation that equates the sum of the height and the perimeter to 96.\answerline
\part[2] Solve the equation in the previous step for one of the variables.\answerline
\part[1] Use the solution in the previous step, write the \emph{volume} of the package as a function of one of the sides. (Note: Volume of a box is \(V=l\times w\times h\))\vspace{0.5in}\answerline
\part[2] Using the function in the previous step and Desmos (or another graphing utility) determine the \emph{maximum} and \emph{minimum} volume possible. Include a sketch or screenshot indicating the max/min. (Note: you should carefully consider what a reasonable domain for this function is!!)

\end{parts}\newpage
\question Reminder: Chris will be hosting 2 review sessions on Tuesday, March 19th at noon and 8pm Arizona time. If you \emph{cannot} make those times, please send him (via email or teams) any questions from the practice exams you would like to see worked out.
\begin{parts}
\part[5] Will you be able to complete at least \emph{one} of the practice exams prior to Tuesday?\answerline
\part[5] Did you know that we record videos of ourselves grading your homework? Please be sure to checkout the ``Feedback'' in D2L Assignments! Some assignments (like this one) which fall on the same week as an exam don't get the videos, but that doesn't mean you can't still ask questions about your results! Feel free to message us on Teams if you need clarification on any Written Homework.
\end{parts}

\end{questions}


\end{document}